\chapter*{Abstract}
\addcontentsline{toc}{chapter}{Abstract}

The OYSTER project has made a cold neutron source operating at $T = \SI{20}{\kelvin}$ available at the Reactor Institute Delft, making colder neutron wavelengths $\lambda$ accessible for neutron scattering instruments. In this thesis, the possibility of realizing a spin-echo modulated small-angle neutron scattering (SEMSANS) instrument at this cold source is explored through a combination of analysis, constrained optimization and Monte Carlo simulations. The goal is to develop a SEMSANS instrument capable of measuring processes in the length range of $\SI{10}{\nano\meter}$ to $\SI{5}{\micro\meter}$ at a single sample to detector distance. Such an instrument could be applied to study colloids such as casein micelles, making it possible to better understand processes like milk turning into yoghurt. 

A simple idealized instrument model is introduced and analysed, taking three different precession device options into consideration as well as pyrolytic graphite and velocity selector monochromators. Six instrument designs combining these components are considered and their spin-echo length measurement ranges are computed using a system of constraints derived from the model. These constraints indicate that precession device limitations as well as detector acceptance angles limit designs with a beam focussed on an effective detector height of $h_e = \SI{10}{\milli\meter}$ and this remains the case after optimizing component positions and source wavelengths using a constrained optimization algorithm. By increasing this effective detector height to $h_e = \SI{30}{\milli\meter}$ and considering a more powerful precession device, it is shown that measurement ranges as great as $\SI{15}{\nano\meter}$ to $\SI{5}{\micro\meter}$ can be reached by again optimizing other parameters.

Monte Carlo simulations of measurements using the original unoptimized instruments are performed on three samples representative of the target length range and it appears that the greater spread in wavelength for instruments with colder wavelengths using a velocity selector makes accurate estimation of the SESANS correlation function through visibility estimation harder, especially for smaller lengths. It is unclear if this can be accounted for using a $\lambda$-spectrum correction of the (analytical) SESANS correlation function or if this is caused by the greater spread in scattering angles.

Altogether, SEMSANS instruments that can measure almost the full target range appear possible, at least when using a pyrolytic graphite monochromator which has a smaller $\lambda$ spread than a velocity selector. More work is needed to better understand limitations and potential corrections related to $\lambda$ spread and acceptance angles. Furthermore, more realistic simulations need to be performed to gain a better understanding of measurement range limitations as well as more accurate estimates of intensity, another important instrument design parameter.

%Additionally, a coarse estimate of instrument intensity is given and a simple model is introduced to illustrate scattering effects occurring at the lower spin-echo limit where a significant fraction of intensity goes undetected, indicating that this might also be a limiting factor for in particular the designs with the colder source wavelength. 
% Optimized variants were computed, showing  

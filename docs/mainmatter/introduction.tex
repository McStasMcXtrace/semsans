\chapter{Introduction}
\label{chapter:introduction}
\label{c1:introduction}
In many industries, there is a need to analyze the structure of materials at length scales ranging from nanometres to micrometers. To do this, different methods exist such as scattering techniques, microscopy, and analysis based on macroscopic properties. What is unique about scattering techniques is that they can probe the bulk of the sample at these length scales \cite{bouwman2021}. This means that instead of looking at the surface of a sample as in microscopy or relying on quantities like density to make statements about the microstructure, scattering techniques directly give information about the inner structure of a sample. 

\section{Small-angle scattering techniques}
\label{c1.1}
Different scattering techniques exist based on X-rays and neutrons. Generally speaking, because these are reciprocal space methods, larger length scales correspond to smaller scattering angles and vice versa. This is the motivation behind small-angle scattering methods like small-angle X-ray scattering (SAXS) and small-angle neutron scattering (SANS) which are used to analyze materials at length scales from $\SI{1}{\nano\meter}$ to a few $\SI{100}{\nano\meter}$. These lengths are significantly larger than for instance crystal lattice constants of a few $\unit{\angstrom}$ that can be determined using techniques like X-ray diffraction that measure wider angles. Past the upper limit of a few $\SI{100}{\nano\meter}$, such small-angle techniques become infeasible for realistic beam sizes and samples, as the deflection of particles in a beam is too slight to learn something about the sample.
\subsection{Neutron spin-echo}
For neutrons, spin polarization can label trajectories across the beam, a technique called neutron spin-echo \cite{mezei1972}. The SANS-derivative techniques spin-echo small-angle neutron scattering (SESANS) \cite{rekveldt1996} and spin-echo modulated small-angle neutron scattering (SEMSANS) \cite{bouwman2009}\cite{bouwman2011} exploit this principle to make smaller angles and correspondingly larger length scales up to about $\SI{10}{\micro\meter}$ accessible using polarized neutrons. Formulated differently, these techniques can be seen to measure a type of real-space density correlation function \cite{krouglov2003} as will also be discussed in Chapter \ref{c2:theory}. This means that these techniques measure sample correlations at a length known as the spin-echo length $\delta$. There will typically only be correlation when $\delta$ is proportional to sample lengths such as particle radii, making measurements easy to interpret by those without a background in scattering theory \cite{bouwman2021}. Expressions for such density correlation functions exist for many different sample models \cite{andersson2008}, making it possible to analyze SESANS and SEMSANS measurements of samples by relating them to known models. Compared to SESANS, SEMSANS has the advantage of performing all polarization manipulations before the sample. This makes SEMSANS compatible with magnetic samples \cite{li2021} in addition to allowing for simpler instrument designs with fewer components \cite{bouwman2009}.

\section{SEMSANS at a cold source}
\label{c1.2}

The basic idea behind SEMSANS as a spin-echo technique is to let a modulated neutron beam scatter of a sample, with modulation frequency $f$ being varied to perform a full measurement. This means that in a classical description, the number of neutrons passing through the sample and potentially scattering will vary across the width or height of the sample approximately like a sine wave. The degree to which this modulation is lost after scattering for each modulation frequency $f$ can be related to the density correlation function of the sample through the spin-echo length $\delta$, which is proportional to both $f$ and wavelength $\lambda$. In SEMSANS instruments, the modulation frequency $f$ is also proportional to neutron wavelength $\lambda$. This is because the devices used to create the modulation rely on the principle of Larmor precession which will be explained in Section \ref{c2.1}. This means that $\delta\propto\lambda^2$. The colder the neutrons are, the greater their $\lambda$, meaning that colder neutrons can be used to measure greater spin-echo lengths at the same modulation frequency $f$. 

SEMSANS instruments have previously been realized at the Hoger Onderwijs Reactor (HOR) at the Reactor Institute Delft (RID) using a thermal neutron source. As part of the OYSTER project (Optimized Yield for Science, Technology, and Education of Radiation), a cold neutron source operating at $T=\SI{20}{\kelvin}$ has been installed and additional improvements have been made to achieve an expected hundredfold improvement in measurement quality or time \cite{OYSTER2008}.

\section{Thesis outline}
\label{c1.3}
In this thesis, the possibility of realizing a SEMSANS instrument at the new cold source at HOR will be explored through a combination of mathematical analysis, constrained optimization, and Monte Carlo simulations using the neutron raytracing software package McStas \cite{willendrup2020}. The goal is to bridge the gap between accessible length ranges of SANS and SEMSANS \cite{bouwman2021} and see if it is possible to measure spin-echo lengths from $\SI{10}{\nano\meter}$ to $\SI{5}{\micro\meter}$ in a single instrument by exploiting the advantage of greater wavelengths. Specifically, the goal is to make the first step towards an instrument design that can measure this full range at a single sample to detector distance. This is an alternative to combining SANS and SEMSANS devices into one as has been proposed before \cite{bouwman2011}\cite{kusmin2017}. An application in the food industry that would benefit from such an instrument is the study of colloids such as casein micelles in (fat-free) milk and derivatives like yogurt and curd. The characteristic length scales of all these would be accessible at an instrument with this target range, facilitating a better understanding of processes like milk turning into yogurt. 

To make the first step towards such a design, this work will investigate whether such an instrument can be realized by assuming an idealized instrument model. This model will be introduced in Chapter \ref{c3} after first summarizing the theory of SEMSANS as a spin-echo technique in Chapter \ref{c2:theory}. Six such idealized instruments, combining different precession device and monochromator types, will be analyzed in Chapter \ref{c4:constraints}, which introduces a constraint model for estimating the spin-echo length measurement ranges of instruments. This model is used to numerically optimize the six types of designs in Chapter \ref{c5:optimization}, giving a better understanding of what is possible with optimal instrument parameters. Through this constraint model, remaining constraints limiting these optimized designs are identified and further optimized designs are presented. Monte Carlo simulations of measurements based on an existing McStas simulation model \cite{bouwman2021b} are performed in Chapter \ref{c6:monte-carlo}. Finally, conclusions about the feasibility of an instrument that can measure spin-echo lengths from $\SI{10}{\nano\meter}$ to $\SI{5}{\micro\meter}$ are presented and recommendations for future work are made.
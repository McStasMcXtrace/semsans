\chapter{Conclusion}
\label{chapter:conclusion}
\label{c7:conclusion}
In this work, the possibility of realizing a SEMSANS instrument at the new cold source at the Hoger Onderwijs Reactor at the Reactor Institute in Delft was explored through a combination of mathematical analysis, constrained optimization and Monte Carlo simulations. After providing an overview of relevant theory and introducing a simplified instrument model, six combinations of three precession device options and two monochromator types with compatible wavelengths $\lambda_0$ were presented. These were analysed in detail and measurements of known samples were simulated using the raytracing software package McStas \cite{willendrup2020}. 

The goal was to explore possibilities for an instrument that could be used to measure processes in the range of $\SI{10}{\nano\meter}$ to $\SI{5}{\micro\meter}$. It was shown by optimizing all available free parameters that when considering a beam that has an effective height $h_e$ of $\SI{10}{\milli\meter}$ on the detector in an instrument no more than about $\SI{5}{\meter}$ in length, it is impossible to measure this full range keeping the distance from sample to detector $L_s$ constant and compatible with a small-angle approximation. The precession device characteristics (field strength limitations and field interface angles) and  $h_e$ were identified to be limiting factors in this initial setting using a constraint model. It was shown that by tripling the effective height to $h_e = \SI{10}{\milli\meter}$ or by using improved foil flippers with a maximal field strength of $B_{max} = \SI{150}{\milli\tesla}$ instead of $\SI{30}{\milli\tesla}$, ranges such as $\SI{40}{\nano\meter}$ to $\SI{5}{\micro\meter}$ or $\SI{30}{\nano\meter}$ to $\SI{5}{\micro\meter}$ become possible. Combining these two enhancements gave an optimized range of $\SI{15}{\nano\meter}$ to $\SI{5}{\micro\meter}$, almost fully covering the target range of $\SI{10}{\nano\meter}$ to $\SI{5}{\micro\meter}$.

Additionally, the effects of using a monochromator with a greater $\Delta\lambda/\lambda_0$ in terms of modulation patterns and scattered intensity were analysed in detail and it appears from simulation results that instruments selecting very cold neutrons (i.e. $\lambda_0 = \SI{8}{\angstrom}$ or above) using a velocity selector might be harder to realize than instruments using a more selective pyroletic graphite monochromator with $\lambda_0 = \SI{4.321}{\angstrom}$ or so due to the greater spread in scattering angles when using a broader spectrum. This complicates the data analysis, making it unclear how to relate the measured visibility changes to sample correlation functions.

In conclusion, it appears that realizing a practical SEMSANS instrument at the cold source that can measure almost the full target range is possible. This requires sufficiently powerful precession devices and large enough beam and detector dimensions. Concrete realizations could use Wollaston prisms and a velocity selector with a colder wavelength of about $\lambda_0 = 9 - 10~\unit{\angstrom}$ or improved foil flippers with a pyrolytic graphite monochromator at a warmer wavelength of about $\lambda_0 = 3 - 4~\unit{\angstrom}$.

\section{Future work}
The purpose of this work was to give a first understanding of what is possible in terms of designing SEMSANS instruments for the cold source. For this reason, a great number of factors were abstracted and the instrument designs analysed were in many ways idealized. For instance, the beam was considered to have a uniform profile with slight divergence which was often neglected and a $\lambda$-spectrum exactly described by a Maxwell-Boltzmann spectrum at $T= \SI{20}{\kelvin}$, with a Gaussian distribution of wavelengths being selected from it using an idealized monochromator. These and other simplifications leave a lot of work to be done in terms of more realistic analysis and simulations taking into consideration the true profile and spectrum of the cold source as well as how the source is in practice collimated. The same goes for simulating the physical precession devices, monochromators, polarizer and the analyzer. Such realistic simulations would also make more accurate intensity estimates possible which is an important step before any potential realization of an instrument. In addition to more realistic simulations, Monte Carlo simulations of optimized designs presented in this work could be performed to learn if it is really possible to measure spin-echo lengths over a range of $\SI{15}{\nano\meter}$ to $\SI{5}{\micro\meter}$. Such additional simulations were not performed due to time constraints.

Finally, more detailed studies of tolerable acceptance angles in SEMSANS as well as the effect of broader $\lambda$-spectra on measurement ranges would make it possible to optimize designs further than was done in this work and make it easier to estimate what can practically be measured using a given design. In addition to implementing the existing correction for limited acceptance angles \cite{kusmin2017}, it might be possible to correct for the broader $\lambda$-spectrum as well by computing the effect on modulation visibility or using Fourier methods, meaning that designs using velocity selectors and broader $\lambda$-spectra might not be further limited but simply require a different treatment than the one presented in this work.

\section{Reproducibility}
All code used in this research is publicly available online in the form of a Github repository\footnote{\url{https://github.com/tbvanderwoude/semsans}}. It contains an overview of how McStas simulations can be reproduced using the provided instrument and component files as well as a basic list of requirements for running the included Python notebooks and other code. This includes an instrument model implemented in Python together with code that can be used to compute optimal designs.
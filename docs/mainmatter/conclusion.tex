\chapter{Conclusion}
\label{chapter:conclusion}
\label{c7:conclusion}
In this work, the possibility of realizing a SEMSANS instrument at the new cold source at the Hoger Onderwijs Reactor at the Reactor Institute in Delft was explored through a combination of mathematical analysis, constrained optimization and Monte Carlo simulations. After providing an overview of relevant theory and introducing a simplified instrument model, six combinations of three precession device options and two monochromator types with compatible wavelengths $\lambda_0$ were presented. These were analysed in detail and measurements of known samples were simulated using the raytracing software package McStas \cite{willendrup2020}. 

The goal was to explore possibilities for an instrument that could be used to measure processes in the range of $10 \unit{\nano\meter}$ to $5 \unit{\micro\meter}$. It was shown by optimizing all available free parameters that when considering a beam size and detector height of $10\unit{\milli\meter}$ in an instrument no more than about $5 \unit\meter$ in length, it is impossible to measure this full range keeping the distance from sample to detector $L_s$ constant. This might be possible however if a larger beam and detector is used or if more powerful precession devices become available as these were identified to be limiting factors both for an initial set of designs and optimized designs.

The effects of using a monochromator with a greater $\Delta\lambda/\lambda_0$ in terms of modulation patterns and scattered intensity were analysed in detail and it appears from simulation results that instruments selecting very cold neutrons (i.e. $\lambda_0 = 8$Å or above) using a velocity selector might be harder to realize than instruments using a more selective pyroletic graphite monochromator with  $\lambda_0 = 4.321$Å or so due to the greater spread in scattering angles when using a broader spectrum. 

\section{Future work}
The purpose of this work was to give a first understanding of what is possible in terms of designing SEMSANS instruments for the cold source. For this reason, a great number of factors were abstracted and the instrument designs analysed were in many ways idealized. For instance, the beam was considered to have a uniform profile with slight divergence which was often neglected and a $\lambda$-spectrum exactly described by a Maxwell-Boltzmann spectrum at $T=20\unit\kelvin$, with a Gaussian distribution of wavelengths being selected from it using an idealized monochromator. These and other simplifications leave a lot of work to be done in terms of more realistic analysis and simulations taking into consideration the true profile and spectrum of the cold source as well as how the source is in practice collimated. The same goes for simulating the physical precession devices, monochromators, polarizer and the analyzer. This would also make more accurate intensity estimates possible which is an important step before any potential realization of an instrument. Lastly, more detailed studies of tolerable acceptance angles in SEMSANS as well as the effect of broader $\lambda$-spectra on measurement ranges would make it possible to optimize designs further than was done in this work and make it easier to estimate what can practically be measured using a given design.
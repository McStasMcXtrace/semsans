\documentclass{article}
\usepackage{graphicx} % Required for inserting images
\usepackage{geometry}
  \newgeometry{vmargin={15mm}, hmargin={12mm,17mm}} 
\title{SEMSANS at the cold neutron source in Delft}
\author{Thom van der Woude}
\date{May 2024}
%Universal prelude for everything sciency, mathy, and what have you
%Make semantic macros, not syntactic macros!
%maths stuff
\usepackage[stretch=10]{microtype}
\usepackage{todonotes}
\usepackage{amsfonts,amsmath, amssymb,amsthm,steinmetz}
\usepackage[utf8]{inputenc}
%nice tables
\usepackage{booktabs}
\usepackage{multicol}
\usepackage{chemfig}
\usepackage{lmodern}
\usepackage{enumitem}
%units for doing a science. allowlitunits means things like 20\milli\meter in math mode without \SI{}{}
\usepackage{siunitx}
%time, you always need time
\usepackage{datetime}
%semi-nice karnaugh maps which can be defined using minterms, maxterms etc.
\usepackage{karnaugh-map}
\newtheorem{theorem}{Theorem}
\usepackage{marginnote}
\usepackage{tikz}
\usepackage{subcaption}
\usepackage[normalem]{ulem}
\usepackage{graphicx} 
\usetikzlibrary{quotes,angles}
%\dv <3
\usepackage{physics}
%bra-ket notation for quantum stuff
\usepackage{braket}
%for surface and volume integrals
\usepackage{esint}
\usepackage{enumitem}
%Specific hackery/macros to save time/make latex more semantic less syntactic
\DeclareMathOperator{\sinc}{sinc}
\newcommand{\conj}[1]{\,\overline{\!{#1}}}
\newcommand{\phasor}[1]{\,\widetilde{\!{#1}}}
\newcommand{\risingfac}[1]{%
	^{\overline{#1}}%
}
%for sets generating groups
\newcommand{\gen}[1]{%
	\langle #1\rangle
}
\newcommand{\fallingfac}[1]{%
	^{\underline{#1}}%
}
%Theorem style for "[Name of famous mathematician]'s Theorem" type of theorems, of which there are a surprising number.
\theoremstyle{named}
\newtheoremstyle{named}{}{}{\itshape}{}{\bfseries}{.}{.5em}{\thmnote{#3's }#1}
\newcommand{\bitvector}[2]{
	\underline{#1} = (#1_{#2-1},#1_{#2-2},...,#1_0)
}
\newcommand{\code}[1]{
	\texttt{#1}
}
\newcommand{\mean}[1]{
	\langle #1 \rangle
}
\usepackage[leqno]{mathtools}
\usepackage{chngcntr}
\counterwithin{equation}{section}
\usepackage[roman, thin, thinp, thinc]{esdiff}

\usepackage{bm}
\usepackage{marginnote}
\usepackage{epigraph}
\usepackage{hyperref}
\usepackage[
backend=biber,
%sorting=ynt
]{biblatex}

\addbibresource{references.bib}
\newcommand{\targetrange}{$10 \unit{\nano\meter}$ to $5 \unit{\micro\meter}$ }
\begin{document}

\maketitle

\section{Introduction}
In the food industry, there is the need to analyse the structure of materials at length scales ranging from nanometres to millimetres. To do this, different methods exist such as scattering techniques, microscopy and analysis based on macroscopic properties. What is unique about scattering techniques however is that they can probe the bulk of the sample at these length scales \cite{bouwman2021}.

% I need to add a better bridge between these two sections, they are both good but lack coupling.

\subsection{Small-angle scattering}

Different scattering techniques exist based on X-rays and neutrons. Generally speaking, because these are reciprocal space methods, larger length-scales correspond to smaller scattering angles and vice versa. This is the motivation behind small-angle scattering methods like Small Angle X-ray Scattering (SAXS) and Small Angle Neutron Scattering (SANS) which are used to analyse materials at length scales from $1 \unit{\nano\meter}$ to a few $100 \unit{\nano\meter}$, which are quite great lengths compared to for instance crystal lattice constants of a few $\unit{Å}$ that can be determined using techniques like X-ray diffraction. Past this upper limit such small-angle techniques become infeasible for realistic beam sizes and samples, as the deflection of particles in a beam is too slight to learn something about the sample.

For neutrons, spin polarization can be used to label trajectories across the beam, a technique called neutron spin echo \cite{mezei1972}. This principle has been applied in the SANS-derivative techniques SESANS \cite{rekveldt1996} and SEMSANS\cite{bouwman2009}\cite{bouwman2011} to make smaller angles and correspondingly the larger length scales up to about $10 \unit{\micro\meter}$ accessible using polarized neutrons. Formulated differently, these techniques can be seen to measure a type of real-space density correlation function $G(\delta)$\cite{krouglov2003}\cite{andersson2008} as will also be discussed in Chapter \ref{c2:theory}. From this

\subsection{SEMSANS at the Reactor Institute Delft}


SEMSANS instruments have previously been realized at the Hoger Onderwijs Reactor (HOR) at the Reactor Institute Delft (RID) using a thermal neutron source. As part of the OYSTER project (Optimized Yield for Science, Technology, and Education of Radiation), a cold neutron source operating at $T=20 \unit{\kelvin}$ has been installed and additional improvements have been made to achieve an expected hundredfold improvement in measurement quality or time.

In this report, the possibility for realizing a SEMSANS instrument at the new cold source at HOR will be explored through a combination of mathematical analysis, constrained optimization and Monte Carlo simulations using raytracing software package McStas \cite{willendrup2020}. An existing McStas simulation model \cite{bouwman2021b} is taken as a starting point and extended to include various design options such as precession devices different from foil flippers. 
The goal is to bridge the gap between accessible length ranges of SANS and SEMSANS \cite{bouwman2021} and see if it is possible to access characteristic lengths from \targetrange by exploiting the advantage of greater wavelengths and without having to combine SANS and SEMSANS devices into one as has been proposed before \cite{bouwman2011}. An application that would benefit from such an instrument is the study of colloids such as casein micelles in milk and derivatives like yoghurt and curd. The characteristic length scales of all these would be accessible at an instrument with this target range, facilitating a better understanding processes like milk turning into yoghurt. 

\section{Theory}
\label{c2:theory}
In this chapter, a review is given of relevant theory related to SEMSANS and the interpretation of measurements. In the first section, the basic principles of SEMSANS as a neutron spin echo technique are explained and the instrument model used throughout this work is introduced, which describes the instrument fully without a sample. In the second section, the interaction of samples with the modulated beam is discussed and the concept of spin-echo length $\delta$ is introduced.

\subsection{Beam modulation through precession in SEMSANS}
Neutrons are uncharged spin $s=\frac{1}{2}$ particles, meaning that their spin can be described as a superposition of spin states $\{\ket{\uparrow}, \ket{\downarrow}\}$ for some chosen axis or equivalently using a spin angular momentum vector $\vec{S}$.
When a neutron passes through a uniform $\vec{B}$-field$\vec{S}$ is in, $\vec{S}$ will precess by a certain angle $\phi$ over time. The frequency at which this occurs is given by
$$\omega = \gamma |B_\perp|$$
with $B_\perp$ the magnetic field strength perpendicular to the plane $\vec{S}$ is in and $\gamma$ the neutron gyromagnetic ratio. 

In addition to spin, a neutron has a wavelength $\lambda$ corresponding to a speed $v = \frac{h}{m\lambda}$, $m$ being the neutron mass. This means that it will pass through a uniform (perpendicular) field $B$ of length $L$ in time $t = \frac{Lm\lambda}{h}$. From this it can be seen that the total precession will be 
$$\phi = \frac{\gamma B L m\lambda}{h} = c\lambda B L$$
with $c = \frac{\gamma m}{h}$ being the Larmor constant \cite{bouwman2021b}.   

\subsubsection{Precession devices}

\subsection{Interpretation of SEMSANS measurements}



%\section{SEMSANS instrument model}
%In this chapter, the used SEMSANS instrument model and all its components with their basic functions as well as parameters is introduced. It is based on an instrument which was previously simulated (CITATION NEEDED) and which is available as NAME in the McStas instrument library. A schematic overview is given in ADD FIGURE HERE. 
%\subsection{Source}

%\section{Source}
%The 
,
% Key challenge: translate this beautiful blob of mathematics and analysis I have in my jupyter notebook to on the one hand a summary and theoretical background and on the other hand a (somewhat, at least in the extent of having fully worked it out myself generally unaware of existing derivations) novel analysis on which
% Another challenge: how to deal with the rather expansive theory with at least 2 main branches which there appears to be? If I give a theoretical background to make this thing readable for other bachelor students then I do sort of have to start at Larmor precession and neutron spins but also at small angle scattering and this is confusing. 

\section{Theory of neutron spin echo and precession in SEMSANS}

\
% Vision: introduce theory in two parts:
% 1. Introduce polarized neutron techniques, Larmor precession and how using polarizer+precession devices+analyzer you can get a modulation pattern on the detector

\section{Real-space interpretation of SEMSANS measurements using spin-echo length}
% 2. Introduce (small angle) scattering techniques and quantities like k, Q, etc. Briefly summarize the greater mathematical framework behind the real-space interpretation of SESANS/SEMSANS. In short, introduce how you can probe a certain spin echo length delta using SEMSANS and why this modulation is relevant in achieving that. 


\section{Instrument design and constraints}
\label{c3:constraints-and-design}

\subsection{Instrument design variants}
% Introduce table of instrument variants based of SEMSANS_Delft instrument with the
% different wavelength operating points and different corresponding monochromators
% Also vary the precession devices to give 9 variants in total or so. 

\subsection{Constraints}
% Discuss sources of constraints, linking them to the theory in previous chapters. Ideally refer to equation numbers to avoid repetition and make it sound really solid. 
% Emphasize that these constraints serve as a starting point for estimating limitations for given instruments

\subsection{Limitations of constraint scheme}
% Discuss things that are missing
% Also very important: discuss the softer constraints: why should the envelope FWHM be 3mm and not 2mm or 4mm? What determines this other than vibes. Indicate how they could be made more/less flexible

\subsection{Computed $\delta$ constraints and final design $\delta$ ranges}
% Give a table of computed constraints per instrument with ideally great names

\subsection{Discussion of designs}
% Here I'll write about clear trends in the tables of limitations and give indications of what is limiting these instruments. It might be interesting to give concrete recommendations: in order to get an even better delta range you could get a different monochromator, boost the B fields etc.


\section{Optimization of instrument parameters}
The problem of designing an instrument suitable for a specific length range like \targetrange subject to the various constraints indicated in the previous chapter can be aptly described as a constrained optimization problem. This does require a formulation of an objective function to maximize and generally speaking this is easier said than done if all factors including instrument cost, intensity etc. are to be taken into account in addition to the $\delta$-range. In this section, the function that is maximized only depends on the $\delta$-range and consequently it should be understood as a speculative exploration of the instrument parameter-space rather than a final optimization of instruments to realize. At best, the presented optimal instruments represent the best one can do in terms of $\delta$-range in the framework of parameters and constraints considered in this research.

\subsection{Two possible $\delta$-range objective functions}
Given the constraints formulated in Chapter \ref{c3:constraints-and-design}, let $(\delta_{i, min},\delta_{i, max})$ be the final $\delta$ interval subject to these constraints and let $(\delta_{t, min},\delta_{t, max})$ be a target interval. If these overlap, let the overlap interval be given by $(\delta_{o, min},\delta_{o, max})$. A first objective function $g_1$ to maximize is ratio of the length of the overlap and the length of the target interval
$$g_1 = \begin{cases}
	\frac{\delta_{o, max}-\delta_{o, min}}{\delta_{t, max} - \delta_{t, min}},\text{ if $\delta$-ranges overlap}\\
	0,\text{ else}
\end{cases}$$
A problem with this objective function is that it doesn't take into consideration that the instrument should perform well across different length ranges and is biased towards the higher $\delta$ values in optimal designs. An intuitive way to optimize for coverage of different length ranges is achieved by looking at these on a logarithmic scale and considering the overlap there, which can be expressed using a second objective function $g_2$ as follows 
$$g_2 = \begin{cases}
	\frac{\ln(\delta_{o, max}/\delta_{o, min})}{\ln(\delta_{t, max}/\delta_{t, min})},\text{ if $\delta$-ranges overlap}\\
	0,\text{ else}
\end{cases}$$

\subsection{Optimization scheme}
Similar to in the previous chapter, all different pairings of precession devices and monochromators are considered. However, $\lambda_0, L_1, L_2, L_s$ are now free parameters with $\lambda_0$ being bounded by the approximate $\lambda$-band of the respective monochromator. 

\subsection{Discussion of optimal instruments}
The resulting optimal instrument parametrizations confirm patterns already visible in the previous chapter. For instance, isosceles triangles seem to be a very bad pairing with a PG monochromator. It can also be seen that instruments with PG monochromators are generally bounded on the upper end by field strength and those with velocity selectors by the narrowing of their envelope caused by wavelength spread. Similarly, almost all instruments are bounded by the condition that at least one period must be visible on the detector, with especially instruments at greater wavelengths $\lambda_0$ suffering from this limitation.  
% add further points that come up, maybe after writing a similar section for designs and their computed limitations

\section{Monte Carlo simulations of designs using McStas}
% after all the analysis and discussion of constraints/optimizing using these, introduce monte carlo simulations as a step closer to a practical instruments and a way to verify identified limitations and see them in play. This last aspect could be an interesting part of the discussion of simulation results as well, you can really point out narrowing modulation envelopes, modulation periods becoming too big for the detecto retcetera. 
% I expect some designs to not be able to measure specific samples at all or barely to the extent that showing them in a plot is a bit silly. 

\subsection{Simulation setup}
% Mention the three samples, perhaps discuss their properties like thickness t etc. besides R. You can just put them in a table quick and dirty.
% Maybe repeat/put an analytical graph of the expected P(\delta) curve here for all 3.
\subsection{Results}
% There will be quite a few graphs here, maybe use Appdendices if it is too much!
\subsection{Discussion}
% Did the simulations match the precomputed limits? Are the limits (to the extent that they are not absolute but rather somewhat chosen heuristically) good enough, too strict? Is it in practice possible to measure even at field strenghts outside of the range?

%\section{Appendix A: Limitations of SEMSANS theory at lower end of $\delta$ range}
%In elastic scattering, the initial and final wave vectors $k_i$, $k_f$ are related through wave vector transfer $Q = |k_f - k_i| = k 2\sin\theta$ with $2\theta$ being the angle between $k_i, k_f$ and $|k_i| = |k_f| = k$ due to elasticity. 

%In normal SEMSANS theory, the small-angle approximation must hold so that $Q_y = k2\theta$ or with $2\theta %\approx \frac{y}{L_s}$, $Q_y = k\frac{y}{L_s}$. Using this approximation, it can be seen that $\phi_t = %\delta Q_y$, which means that the modulation amplitude $A(\delta)$ is related to
%$$G(\delta) = \frac{1}{\sigma k^2}\int_{Q_{x,min}}^{Q_{x,max}}\int_{Q_{y,min}}^{Q_{y,max}}\dfrac{d\sigma}{d\Omega}(Q)cos(\delta Q_y)dQ_ydQ_x$$
%by the formula
%$$A(\delta) = e^{\sigma [G(\delta) - 1]}$$
%It can be shown that the generally correct expression for $Q_y$ is 
%$$Q_y = 2k\sin(\frac{\arctan\frac{y}{L_s}}{2})$$
%This can be simplified to the somewhat lengthy expression
%$$Q_y = 2k \frac{\frac{y}{L_s}}{\sqrt{2}\sqrt{\frac{y^2}{L_s^2} + 1 }\sqrt{\frac{1}{\sqrt{\frac{y^2}{L_s^2} + 1}}+1}}$$
%Considering again small $\frac{y}{L_s}$, it can be derived using a Taylor expansion that
%$$Q_y \approx  k(\frac{y}{L_s} - \frac{3y^3}{8L_s^3})$$
%So a more accurate relation between $\phi_t$ and $Q, \delta_y$ is given by $$\phi_t = \frac{\delta Q_y}{1 - \frac{3y^2}{8L_s^2}}$$
% This appears to give an indication of up to which angles the SEMSANS theory gives a correct description of the amplitude decrease.




\printbibliography
\end{document}
